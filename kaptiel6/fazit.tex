\chapter{Fazit}

Legal Tech ist nicht nur ein Trendthema. Vielmehr konnte aufgezeigt werden, dass Legal Tech dazu in der Lage ist den Rechtssektor und den dazu gehörigen Markt tiefgreifend zu verändern.
Die zu Verfügung stehenden Anwendung im Legal Tech Bereich sind mehr als umfangreich.
Das Potential ist noch bei weitem nicht vom Markt ausgeschöpft, wie in ,,\nameref{Einsatzbereiche von Legal Tech}'' beschrieben, sind
die Möglichkeiten, Juristen technisch zu Unterstützen, sehr weit gestreut. Ob bei der einfachen Dokumentenerstellung oder bei der
Lösung komplexer Rechtsfällen, der Markt bietet zumindest theoretisch eine Lösung für jedes Problem. Das wahre Hindernis des völligen Durchbruchs von Legal Techs
liegt in den Nutzern. Diese sind, zumindest im aktuellen Stand, laut dem Star-Report\footfullcite{STARStatistik} noch nicht bereit,
tiefgreifende strukturelle Veränderungen vor zu nehmen. Stand heute, wird Legal Tech von Anwälten hauptsächlich nur zur Unterstützung der bereits
vorhandenen Prozesse genutzt. Aktuell gibt es erst sehr wenig spezielle Legal Tech-Unternehmen. Leider verändern nur sehr wenige Rechtsanwaltskanzleien ihre Abläufe und passen diese den vorhandenen Technologien an. Des weiteren sind die rechtlichen Hindernisse in der Bundesrepublik Deutschland noch nicht geklärt. In vielen Fällen ist eine sorgenfreie Nutzung von höheren Legal Tech-Stufen noch nicht möglich.
Das führt zur Antwort der Leitfrage dieser Arbeit:
Hat Legal Tech das Potential, den Markt zu verändern und falls ja, wie weit ist ein Fortbestehen der alten Methoden neben Legal Tech möglich?
Nach den aktuellen Auffassungen hat Legal Tech durchaus das Potential, tiefgreifende Veränderungen am Markt zu erschaffen. Es ist in der Lage,
denjenigen, die es ausgereift nutzen, einen entschiedenen Vorteil am Markt zu generieren. Dennoch ist das Potential noch lange nicht so ausgeschöpft,
wie man es sich wünschen würde. Die Herausforderung liegt darin, den Markt so zugänglich zu machen, dass selbst ungeübte Nutzer einen einfachen
Einstieg haben. Sollte das gelingen, ist ein Fortbestehen der alten Methoden, die ohne Legal Tech arbeiten, unwahrscheinlich.