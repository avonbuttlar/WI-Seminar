\chapter{Rechtliche Implekationen}
Ob die Nutzung von Legal Tech-Anwendungen rechtlich zulässig sind muss vorab geklärt sein. Das regelt in der Bundesrepublik Deutschland unter Anderem das Rechtsdienstleistungsgesetz. Im Folgenden soll aufgezeigt werden, in welchen Punkten Legal Tech sich noch in rechtlichen Grauzonen befindet. 

\section{Rechtsdienstleistungsgesetz (RDG)}
Das Rechtsdienstleistungsgesetz regelt in § 1 Absatz 1 die Befugnis, in der Bundesrepublik Deutschland außergerichtliche Rechtsdienstleistungen zu erbringen. Es dient dazu, die Rechtsuchenden, den Rechtsverkehr und die Rechtsordnung vor unqualifizierten Rechtsdienstleistungen zu schützen. In diesem Zusammenhang stellt sich insbesondere für Legal Tech-Anwendungen, die für die Vereinfachung von juristischen Tätigkeiten eingesetzt werden, die Frage, ob die rechtlichen Vorgaben des RDG eingehalten werden müssen. Nach § 2 RDG ist eine Rechtsdienstleistung jede Tätigkeit in konkreten fremden Angelegenheiten, sobald sie eine rechtliche Prüfung des Einzelfalls erfordert. In Anwendung dessen ist streitig, ob eine Software überhaupt eine Rechtsdienstleistung im Sinne von § 2 RDG erbringen kann. Dadurch, dass der Nutzer nur mit einer Software zu tun hat, ist fraglich, ob überhaupt der Anwendungsbereich des Rechtsdienstleistungsgesetz eröffnet ist.\footfullcite [S. 45]{LegalRobots}

Hierbei wird man auf die genauen Umstände des Einzelfalls abstellen müssen. Für die abstrakte Erstellung von Vertragsdokumenten durch eine Software hat der Bundesgerichtshof bereits eine erste Grundsatzentscheidung getroffen. Zwar sei das RDG grundsätzlich anwendbar, jedoch sei durch die streitgegenständliche Software keine Lösung einer konkreten fremdem Angelegenheit angestrebt. [ BGH, Urteil vom 09.09.2021, Az.: I ZR 113/20.] Begründet wurde dies mit dem Umstand, dass der Anwender mithilfe des Programms selbst ,,bloß'' ein Rechtsdokument erstellen kann. Das Programm fungiert dabei nur als Hilfsmittels vergleichbar mit einem Formularhandbuch. Eine eigene Rechtsangelegenheit in eigener Verantwortung, ohne dass er eine rechtliche Beratung bei der Formulierung des Rechtsdokuments stattfindet, war nicht gegeben. Im Ergebnis verneinte der Bundesgerichtshof deshalb einen Verstoß gegen des Rechtsdienstleistungsgesetz.

In Zukunft wird man abwarten müssen, inwieweit der Bundesgerichtshof die konkreten Angebote von Legal Tech- Unternehmen einordnet. Für mehr Rechtssicherheit wäre aufgrund der Auslegungsfähigkeit des geltenden Rechts eine Gesetzesreform des Gesetzgebers sehr begrüßenswert.\footfullcite [S. 35]{Engelhardt}

\section{Haftung}

Im Anschluss an die vorangestellten Überlegungen stellt sich eine wichtige Folgefragen bezüglich der Haftung von Anbietern von Legal Tech-Software. Während Rechtsanwälte gemäß § 51 Absatz 1 Satz 1 BORA verpflichtet sind, eine Berufshaftpflichtversicherung zur Deckung der sich aus seiner Berufstätigkeit ergebenden Haftpflichtgefahren für Vermögensschäden abzuschließen und die Versicherung während der Dauer seiner Zulassung aufrechtzuerhalten, gibt es eine solche Verpflichtung nicht für sonstige Anbieter von „Rechtsdienstleistungen“. Die Anbieter haften dann je nach Ausgestaltung der Rechtsform nur in Höhe der Einlage oder mit dem privaten Vermögen. Im Angesicht der zu meist geringen Streitwerte von Streitigkeiten, die über Legal Tech-Unternehmen abgewickelt werden, scheint das auf dem ersten Blick auch nicht notwendig. Da sich diese Fälle jedoch meist auf mehrere tausende Geschädigte erstreckt, könnte aus auch hier angezeigt sein, eine entsprechende Verpflichtung einer Haftpflichtversicherung auch auf den Bereich der Legal Tech-Unternehmen auszuweiten. Eine Exkulpation des Anbieters wird in den wenigsten Fällen gelingen, da Fehler in der Programmierung immer den Anbieter selbst zuzurechnen sein werden. Sollte auf künstliche Intelligenz basierende Entscheidung getroffen werden, so würde wohl der ebenso der Fehler der künstlichen Intelligenz dem Anbieter nach § 278 BGB (analog) zugerechnet werden.

Gänzlich ungeklärt sind die Auswirkungen auf die Pflichterfüllung von Organen in Unternehmen, die sich auf Ergebnisse einer Legal Tech-Software verlassen und nicht den Rechtsrat von Rechtsanwälten oder Rechtsabteilungen einholen.\footfullcite[S.47]{LegalRobots} Gerade auch im Hinblick auf die Regressmöglichkeit bei den Berufshaftpflichtversicherungen von Rechtsanwälten, kann es sinnvoll sein, das Ergebnis über einen Rechtsanwalt absichern zu lassen, um nicht selbst eine Pflichtverletzung im Rahmen der Vorstands/Geschäftsführertätigkeit zu begehen.