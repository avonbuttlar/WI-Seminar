\chapter{Einsatzbereiche von Legal Tech} \label{Einsatzbereiche von Legal Tech}
Im folgenden Abschnitt wird Bezug auf die einzelnen Bereiche des Legal Tech genommen. Ferner wird beschrieben wie und wo Legal Tech Personen im traditionellen Rechtssektor unterstützt. Jeder Unterpunkt beschreibt eine bestimmte Art des Legal Tech's.

\section{Dokumente}
In den folgenden Unterpunkten sind Legal Tech-Einsatzbereiche zusammengefasst, die den Juristen bei der Bewältigung ihres Alltags helfen. Allgemein kann man diese unter \nameref{Legal Tech 1.0} zusammenfassen.
\subsection{Dokumentenerstellung}
Die offensichtlichste und einfachste Weise, wie Technologie einen Anwalt oder einen Richter unterstützen kann, ist in der Dokumentenerstellung. Es liegt in der Natur der Sache, dass im Rechtswesen viele und vor allem lange Schriftstücke erstellt und verschickt werden. Hierfür benötigt es Software, die es den Juristen ermöglicht schnell und fehlerfrei Schriftstücke zu erstellen. Oft reichen in größeren Kanzleien nicht nur einfache Word-Vorlagen. Da Schriftsätze oft über mehr als 100 Seiten verfügen und penible Genauigkeit erfordern, wird hier vom Markt speziell für den Rechtssektor ausgelegte Software bereit gestellt. Diese verfügt dann normalerweise zum einen über Funktionen zum erstellen solcher Dokumente, zum anderen auch über Funktionen zum kreieren von Musterstücken, die dann zur Massenproduktion eingesetzt werden können. Hierfür wird meistens eine Kombination aus Microsoft Word und dem in der Kanzlei vorhandenen Aktenverwaltungssoftware genutzt.

 \subsection{Dokumentenanalyse}
Zum Beruf eines Juristen gehört es nicht nur selber Schriftstücke zu erstellen, sondern auch auf solche der Gegenseite zu antworten. Da wie bereits erwähnt diese Schriftstücke üblicherweise sehr lang sind, würde es auf dem traditionellen Weg sehr lange dauern, alle diese Schriftstücke vollumfänglich zu lesen. Hierfür bietet der Legal Tech-Markt auch Lösungen: Programme, die dazu geschaffen wurden, zunächst den Adressat / Fall des juristische Werks herauszufinden und dann zu analysieren und deren Kernaussage heraus zu filtern. Hierbei wird zum Beispiel auf fehlende Kriterien, fehlerhafte Klauseln oder generelle Risiken geprüft. Das bekannteste Programm zur Dokumentenanalyse ist Juristische Textanalyse der DATEV eG.

\subsection{Datenbanken und Wissensmanagement-Systeme}
Zur Klärung des Sachverhaltes und Herausfiltern etwaige Ansprüche benötigt es juristischer Kenntnisse. Da diese aber sehr umfangreich und niemals vollumfänglich präsent seien können, bedarf es auch hier an Unterstützung. Früher war hierfür eine umfangreiche Bibliothek mit abermals vielen Aufzeichnungen nötig. Lange Recherchen nach den richtigen Schriftstücken gehörten zum juristischen Beruf dazu. Dafür bietet der Legal Tech-Sektor Abhilfe. Datenbanken und Wissensmanagment-Systeme, die den Juristen in seiner Arbeit unterstützen sollen, werden hierfür angeboten. Es werden jegliche Urteile, Gesetze, Kommentare und Aufsätze zu einem bestimmten Thema gesammelt und dem Juristen auf Knopfdruck angeboten. Dadurch lässt sich dieser Teil der Arbeit äußerst effizient gestalten. Vor allem im anglo-amerikanischen Rechtskreis sind diese Datenbanken essenziell. Dort ist das Rechtssystem auf Fallrecht aufgebaut, das heißt die primäre Rechtsquelle ist nicht wie in Deutschland die generellen Gesetzte, sondern die richterlichen Entscheidungen konkreter Fälle. Deshalb müssen dort für Rechtsstreite so genannte Präzedenzfälle gesucht werden, die auf den eigenen Rechtsstreit anwendbar sind. Im europäischen Raum gilt kodifiziertes Recht, dass heißt es wird nach dem Gesetz und nicht nach vorher verhandelten Fällen entschieden. Somit müssen für diese Anwendungsfälle verschiedene Datenbanken angeboten werden. Die bekannteste Deutsche Rechtsdatenbank gehört dem Beck Verlag und heißt beck-online.

\section{Vermittlung}
Vermittlung und Findung des richtigen Partners ist immer der erste wichtige Schritt zur Lösung eines Problems.
In diesem Abschnitt werden Techniken genannt, die alle am Rechtssektor teilnehmenden Personen bei der Findung des richtigen Partners unterstützen sollen.
\subsection{Plattformen für Anwaltsleistungen} \label{Plattformen}
Durch das Internet werden viele Dienstleistungen, die früher lokal (,,vom nächst Besten'') erledigt wurden, nicht mehr unbedacht vergeben. Selbes gilt auch für den Rechtssektor. Auch hier wird mittlerweile vom Mandanten bewertet, verglichen und abgewogen. ,,Durch Abschaffung der Singularzulassung der Rechtsanwäte [...] [Gibt es] die Möglichkeit, anwaltliche Dienste ohne örtliche Beschränkungen anbieten zu können [...]''. \footfullcite[S. 11]{LegalRobots} Im Folgenden werden verschiedene Plattformen beschrieben, durch die sich der anwaltliche Beruf verändert hat.
\subsection{Anwaltsportale und Anwaltssuchmaschinen} \label{Anwaltsportale}
Anwaltsportale und Anwaltssuchmaschinen sind, wie es der Name schon sagt, Plattformen um den passenden Anwalt zu finden. Der Anwalt selber kann ein Profil anlegen, angeben welche Rechtsgebiete er abdeckt und eigene Veröffentlichungen zu bestimmten Themen bekannt geben. User, die auf der Suche nach einem Anwalt sind, können eben solche finden und üblicherweise Bewertungen über jene abgeben. Diese Bewertungen wiederrum beeinflussen dann das Standing und Ranking des Anwalts. Der bekannteste Vertreter dieser Portale ist Anwalt.de.

\subsection{Marktplätze}
Mandanten, die Portale wie in dem Kapitel ,,\nameref{Anwaltsportale}'' beschrieben nutzen, haben oft ein ausgereiftes oder kompliziertes Rechtsproblem und suchen intensive Betreuung und Beratung. Das trifft aber nicht auf jedes Rechtsproblem zu. Oft möchten Privatpersonen nur eine rechtliche Einschätzung zu bestimmten Sachverhalten und nicht direkt ein Mandantenverhältnis zu einem Anwalt. Hierfür bieten verschiedene Plattformen eine Art Marktplatz für Rechtsberatung. Der Fragende stellt eine Art Angebot ein. Er beschreibt, was sein rechtliches Problem ist und in welchem Punkt er Hilfe benötigt. Optional kann er auch direkt einen Betrag seinem Angebot mitgeben, den er bereit wäre zu zahlen. Dann können von der Plattform bestätigte Anwälte sich diesem Problem annehmen und entweder den angebotenen Geldbetrag akzeptieren oder ein (Gegen)Vorschlag für die vom Anwalt erstellte Lösung des Problems anbieten. Bekannte Vertreter dieser Art von Marktplatz sind frage-einen-anwalt.de oder advocado.

\subsection{Terminvertretung} \label{Terminvertetung}
In der deutschen Zivilprozessordnung (ZPO) ist geregelt, welches Gericht für welche Verhandlungen zuständig ist. In §12 Allgemeiner Gerichtsstand der ZPO ist folgendes geregelt : ,,Das Gericht, bei dem eine Person ihren allgemeinen Gerichtsstand hat, ist für alle gegen sie zu erhebenden Klagen zuständig, sofern nicht für eine Klage ein ausschließlicher Gerichtsstand begründet ist.''. Dabei wird der allgemeine Gerichtsstand des Wohnsitzes (§13 ZPO)wie folgt definiert : ,,Der allgemeine Gerichtsstand einer Person wird durch den Wohnsitz bestimmt.''. Daraus folgt, dass ein Kläger für Verhandlungen immer zum Gericht des Beklagten kommen muss. Durch die Abschaffung der in \nameref{Plattformen} beschriebene Singularzulassung, kann es aber durchaus vorkommen, dass ein Anwalt aus Konstanz einen Mandanten in Kiel vertritt. Unter Berücksichtigung, dass Anwälte üblicherweise einen im Verhältnis zur restlichen Bevölkerung hohen Stundensatz haben, wäre es in diesem Fall wirtschaftlich gesehen nicht sinnvoll, dass der Anwalt selber vor Gericht erscheint. Hierfür können Anwälte andere Anwälte als Terminvertretung beauftragen. Vor der Zeit von Legal Tech war hierfür ein ausgereiftes Netzwerk notwendig, da weder Erfahrungen noch Kontakte zuverlässig öffentlich geteilt werden konnten. Diese Lücke hat Legal Tech aber durch so genannte Terminvertretungsportale geschlossen. Anwälte können hier entweder ihre Dienste für eine Anzahl von Gerichten anbieten oder aber sich für ihre Fälle passende Terminvertretungen suchen. Beispiel für ein solches Portal ist terminsvertreter.com.

\section{Automatisierte Bearbeitung}
Für alle Branchen ist es wichtig, die Effizienz zu steigern. Dies ist am einfachsten zu bewältigen, indem wiederholende Aufgaben minimiert oder standardisiert werden. Dies gilt auch für den Rechtssektor.
Unter ,,Automatisierte Bearbeitung'' werden alle Legal Tech-Bereiche zusammengefasst, die (teilweise) den Juristen oder Anwalt ersetzen sollen.

\subsection{Smart Contracts}
\label{SmartContracts}
Unter Smart Contracts werden digitale oder digitalisierte Verträge verstanden, die sich selber vollziehen. Die Vertragsbedingungen werden dabei direkt in einer Programmiersprache geschrieben. Es ist also kein Vollstrecker im herkömmlichen Sinne mehr nötig. Die zu vollstreckenden Vertragsklauseln werden abstrakt gesprochen von einer Technologie, oder zumindest nicht von einem Menschen vollzogen. Aufgrund dieser Automatisierung ist keine weitere Instanz nötig. Beispielsweise könnte ein Testament durch einen Smart Contract realisiert werden. Dann wären innerhalb Deutschlands, sofern die rechtliche Grundlage geschaffen ist, keine Testamentsvollstrecker mehr nötig. Der Smart Contract würde vollautomatisiert vollstreckt werden, sobald die Behörde den Tod eines bestimmten Menschen feststellt. Dadurch lassen sich typische Fehlerquellen ausschließen.
\subsubsection{Blockchain}
\nameref{SmartContracts} setzen eine gewisse Unveränderlichkeit voraus. Bei einem normalen Vertrag können alle Bedingungen mit Abstimmung des Vertragspartners geändert werden. Ob diese Bedingungen vollzogen wurden, können nur die beiden Parteien oder im Zweifelsfall ein Gericht entscheiden. Bei einem Smart Contract ist das ein Problem. Um dem entgegen zu wirken sollen Smart Contracts in eine Blockchain eingebracht werden. Eine Blockchain ist ein neuartiger Datenbanktyp, dessen Hauptmerkmal Dezentralisierung ist. Dadurch werden die Verträge durch Manipulation der beiden Parteien oder Dritter geschützt. Ein Beispiel für die Kombination von Smart Contract und Blockchain könnte mit der Cryptowährung Ethereum (die bereits auf der Blockchain-Technologe gebaut ist) realisiert werden. So könnte man beispielsweise einen Vertrag aufsetzen, der eine Dienstleistung enthält, die mit einem bestimmten Betrag an Etherium an eine bestimmte Adresse bezahlt wird. Sobald dann eine dritte Schnittstelle die Erbringung der Dienstleistung bestätigt, wird automatisch der vorher definierte Betrag an Etherium an die Adresse überwiesen. 
\subsection{Rechts-Generatoren}
\label{Rechts-Generatoren}
Rechts-Generatoren sind einer der ersten Schritte des \nameref{Legal Tech 2.0}. Hierbei soll der potentielle Mandant vorab einen Fragenkatalog zum Sachverhalt beantworten. Die Antworten des Mandanten werden dann durch eine Software geprüft und es findet eine automatische rechtliche Beurteilung statt. Sollte ein Rechtsanspruch vorliegen, so erstellt die Software üblicherweise auch automatisch ein passendes Schriftstück, das dann nur noch von einem Anwalt verschickt werden muss. Beispiele hierfür sind Start-Ups wie MyFlyRight oder RefundRebel. Diese nutzen solche Rechts-Generatoren, um einen Markt zu bedienen, der auf herkömmlicher Weise nicht wirtschaftlich bedient werden könnte. Die Streitwerte wären zu gering, dass ein Anwalt sich damit auseinander setzen könnte und die anwaltlichen Kosten den durch die Klage erstreitbaren Mehrwert übersteigen würden. \footfullcite[S. 27]{LegalRobots}. Beispielsweise nutzt RefundRebel solche Rechts-Generatoren, um automatisch das Erstattungsformular der Deutschen Bahn AG auszufüllen.

\subsection{Chatbots}
Einen Chatbot kann man beschreiben, als ein softwareunterstütztes Dialogsystem, das nach bestimmten Regeln oder mit Hilfe künstlicher Intelligenz mit Menschen interagiert, sei es durch Texteingabe, Sprache oder Gestik. Der Vorteil von Chatbots ist, dass man dem potentiellen Mandanten eine 24/7 Versorgen vortäuschen kann. Chatbots ermöglichen es, jederzeit auf Anfragen zu reagieren, diese einzuordnen und in vielversprechenden Fällen tieferen Kontakt zu pflegen oder einen Menschen hinzu zu ziehen. Chatbots werden dabei oft in Kombination mit \nameref{Rechts-Generatoren} eingesetzt. Man könnte Sie auch als Erweiterung der \nameref{Rechts-Generatoren} sehen. Bekannter Nutzer der Chatbots ist beispielsweise DoNotPlay. Diese haben in ihrem Sozialen Netzwerk Chatbots integriert.
\footfullcite{LegalTech}
\subsection{Robo-Judge}
Nicht nur Anwälte sollen von der Automatisierung durch Legal Tech profitieren. In verschiedenen Ländern (darunter Lettland und die Vereinigten Staaten von Amerika) kommen bereits sogenannte Robo-Judges zum Einsatz. Robo-Judges sind Programme, die entweder über Fallentscheidung oder mithilfe von künstlicher Intelligenz über Rechtsstreitigkeiten entscheiden sollen. In den USA wird diese Art von Technik genutzt, um Strafmaß und Kaution fest zu legen. Lettland hingegen geht einen Schritt weiter. Dort sollen Robo-Judges alle Streitigkeiten mit einem Streitwerte unter 7000€ entscheiden. \footfullcite{Engelhardt} Besonders im Vertrags- und Schadensersatzrecht kann der Robo-Judge angewendet werden, da hier oft die Rechtsverhältnisse klar und unmissverständlich sind.  