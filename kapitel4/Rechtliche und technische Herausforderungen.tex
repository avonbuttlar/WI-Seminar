\chapter{Auswirkungen}
Legal Tech hat durch seine Innovationen verschiedene Auswirkungen auf den Beruf der Juristen. Vieles muss oder besser gesagt soll sich an die neue Technik anpassen. Dies trifft aber nicht nur die Juristen. Auch Privatpersonen erleben durch das Aufleben von Legal Tech-Unternehmen neue Möglichkeiten, von ihrem Recht Gebrauch zu machen. Dies führt dazu, dass Privatpersonen eher gewillt sind, vor Gericht zu ziehen beziehungsweise ihre Rechte geltend zu machen. \footfullcite{Friedmann}
\section{Im privaten Sektor}
Durch Legal Tech hat sich das Geltendmachen von Ansprüchen im Zivilrecht in vielen verschiedenen Aspekten geändert. 
\subsection{Zugang zum Recht}
In ,,\nameref{Rechts-Generatoren}'' wird bereits beschrieben, dass Legal Tech zu einem erleichterten Zugang zur Rechtsberatung führt. Das ist gesellschaftlich gesehen die wichtigste Auswirkung von Legal Tech. Gegen Handlungen, gegen die es früher unwirtschaftlich war vor zu gehen, kann jetzt Widerspruch eingelegt werden. Legal Tech bietet die technische Möglichkeit, eine Vielzahl von gleich gelagerten Fällen mit adäquatem Aufwand abzuwickeln. Beispiele hierfür sind die bereits genannten Tools, die einem  bei Verspätung eines Zuges helfen, etwaige Rückerstattungsansprüche zu erheben.
Für den erleichterten Zugang sorgen aber nicht nur die ,,\nameref{Rechts-Generatoren}''. Auch ,,\nameref{Plattformen}'' helfen dem Bürger, in jedem Fall den besten Anwalt für seine Rechtsprobleme zu finden. 

\subsection{Kauf und Abtretung von Ansprüchen}
,,\nameref{Rechts-Generatoren}'' haben auch dazu geführt, dass sich eine ganz neue Art von Geschäftsmodell entwickelt hat. Legal Tech-Unternehmen vertreten nicht ihren Mandanten vor Gericht, sondern kaufen dem Mandanten für einen vorab festgelegten Preis seine Ansprüche ab und gehen selbst vor Gericht. Dieser Ansatz soll den Verbraucher und das damit zusammenhängende Verbraucherrecht stärken. Zunächst hilft es dem Verbraucher schneller seinen entstandenen Schaden (zumindest teilweise) zurück zu erhalten. Dieser gibt seine Daten und Ansprüche bei einer Firma an, diese zahlt ihm dann einen gewissen Prozentsatz des Streitwertes für seine Ansprüche. Der Verbraucher kommt somit schnell zu seinem Schadensersatz und muss keine weiteren Bemühungen machen. Das Risiko des Scheiterns der Geltendmachung der Ansprüche wird an die Firma abgetreten. Diese Firmen vertreten dann oft eine Vielzahl von Mandanten (beziehungsweise deren Ansprüche) in der selben Sache. Das ist auch aus Sicht der Prozessökonomie begrüßenswert. Da Gerichte mit der steigenden Automatisierung und damit erhöhten Klagewellen überlastet werden, können Legal Tech-Unternehmen die Ansprüche von vielen Verbrauchen gesammelt geltend machen und nicht für jeden einzelnen Anspruch ein eigenes Verfahren eröffnen. Die Unternehmen selbst haben ein Interesse daran, Fälle, die von der Norm abweichen, nicht weiter vor Gericht zu verhandeln da wie bereits erwähnt typischerweise der Streitwert gering ist \footfullcite{Engelhardt}
\section{In den Kanzleien}
Auch die Arbeit in den Kanzleien hat sich durch die in ,,\nameref{Einsatzbereiche von Legal Tech}'' genannten Methoden verändert. Im Folgenden werden ein Teil der Veränderungen beschrieben und erklärt.
\subsection{Prozessoptimierung}
Mit Prozessoptimierung ist eine geplante Veränderung der Prozesse, um Effektivität und Effizienz der ausgeführten Aktivitäten zu verbessern, gemeint. Legal Tech zwingt Kanzleien und Unternehmen dazu, in allen internen Abläufen und Prozessen zu optimieren und standardisieren. Das ist am Markt die Voraussetzung, um weiterhin gewinnbringend Rechtsdienstleistung anbieten zu können. Aus diesem Grund fördert Legal Tech die Spezialisierung der Kanzleien auf wenige Rechtsgebiete, um möglichst hohe Übereinstimmung und damit Standardisierung zu ermöglichen. Indirekt verbessert damit Legal Tech die Rechtsdienstleistung als solche. Denn, wenn Kanzleien sich nicht auf viele verschiedene Themen konzentrieren, sondern sich auf wenige Spezialisieren, werden somit Experten und damit bessere Rechtsdienstleistungen geschaffen.
\subsection{Arbeit des Anwalts}
Durch den steigenden Legal Tech Anteil im Rechtssektor werden Juristen dazu gezwungen, sich mit dieser Technik auseinander zu setzen. Dies hat zur Folge, dass auch in der Ausbildung immer mehr darauf geachtet wird, dass Anwälte nicht nur juristisch, sondern auch technisch auf dem höchsten Niveau sind. In Deutschland wird beispielsweise an den Universitäten Hamburg, Münster als auch an der LMU München Legal Tech als Fach angeboten \footfullcite{Engelhardt}. 
\subsection{Veränderung des Vergütungsmodell}
Durch Legal Tech wird das herkömmliche Vergütungsmodell, also die Abrechnung auf Basis von Zeiteinheiten, infrage gestellt. Legal Tech-Unternehmen oder Kanzleien, die sich auf die vollautomatisierte Bearbeitung von Rechtsfällen spezialisiert haben, können nicht nach so genannten Billabale Hours abrechnen. Diese wären nämlich dann verhältnismäßig mäßig gering. Die hat zur Folge, dass Kanzleien nicht mehr auf Stundenbasis mit dem Mandanten abrechnen, sondern vorab eine Gebühr die auf einem Prozentsatz des Streitwertes beruht vereinbaren oder aber einen Festpreis für eine bestimmte Leistung festlegen \footfullcite{STARStatistik}. 

\section{In der Judikativen}
Auch die öffentlichen Stellen mussten sich an die durch Legal Tech angestoßenen technischen Veränderungen anpassen.
\subsection{Digitaler Schriftverkehr} \label{DigitalerSchriftverkehr}
Seitdem 2013 das Gesetz zur Förderung des elektronischen Rechtsverkehrs mit Gerichten (ERV-Gesetz) verabschiedet wurde, ist es möglich, mit Zivil-, Arbeits-, Sozial-, Verwaltungs- und Finanzgerichten digital zu kommunizieren. Spätestens seit 2022 müssen alle am Rechtsverkehr teilnehmenden Personen und damit insbesondere Notare, Rechtsanwälte und Unternehmensjuristen verpflichtend am elektronischen Rechtsverkehr teilnehmen (§12 Handelsgesetzbuch). Hierfür wurde das besondere elektronische Anwaltspostfach (beA) eingerichtet. Über beA können Rechtsanwälte und Gerichte untereinander Schriftstücke zustellen \footfullcite[S.17]{LegalRobots}.

\subsection{Elektronische Akte}
Zu der in ,,\nameref{DigitalerSchriftverkehr}'' genannten Pflicht zum elektronischen Rechtsverkehr hat sich Deutschland zusätzlich zu einer  elektronischen Akte in den Behörden verpflichtet. Spätestens Anfang 2026 soll die gesamte Justiz mit einer elektronischen Akte ausgestattet sein. Dies soll nicht nur die Umwelt schonen, sondern auch die Prozesse und Auskünfte an Prozessbeteiligte beschleunigen \footfullcite[S.17]{LegalRobots}

\subsection{Online-Prozesse}
Im Jahr 2002 wurde §128a der ZPO verabschiedet. Dieser regelt, dass Verhandlungen im deutschen Zivilprozessrecht auch in Bild- und Tonübertragung verhandelt werden können. Spätestens seit der jüngsten Pandemie sind auch die meisten Gerichte darauf vorbereitet. \footfullcite{onlineVerhandlungen} Somit kann durch Legal Tech der in ,,\nameref{Terminvertetung}'' beschrieben Unwirtschaftlichkeit durch lange Anfahrtsstrecken Einheit geboten werden.

