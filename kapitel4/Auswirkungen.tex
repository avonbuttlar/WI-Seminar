\chapter{Auswirkungen}

\section{Im privaten Sektor}
\subsection{Zugang zum Recht}
In \nameref{Rechts-Generatoren} wird bereits beschrieben, dass Legal Tech zu einem erleichterten Zugang zur Rechtsberatung führt. Das ist gesellschaftlich gesehen die wichtigste Auswirkung von Legal Tech. Gegen Handlungen, gegen die es früher Unwirtschaftlich war vor zu gehen, kann jetzt Widerspruch eingelegt werden. Legal Tech bietet die technische Möglichkeit, eine Vielzahl von gleich gelagerten Fällen mit adäquatem Aufwand abzuwickeln. Beispiele hierfür sind die bereits genannten Tools, die einem  bei Verspätung eines Zuges helfen, etwaige Rückerstattungsansprüche zu erheben.
Für den erleichterten Zugang sorgen aber nicht nur die \nameref{Rechts-Generatoren}. Auch \nameref{Plattformen} helfen dem Bürger, in jedem Fall den besten Anwalt für seine Zwecke zu finden. 

\subsection{Kauf und Abtretung von Ansprüchen}
\nameref{Rechts-Generatoren} haben auch dazu geführt, dass sich eine ganz neue Art von Geschäftsmodell entwickelt hat. Legal Tech-Unternehmen vertreten nicht ihren Mandanten vor Gericht, sondern kaufen dem Mandanten für einen vorab festgelegten Preis seine Ansprüche ab und gehen selbst vor Gericht. Dieser Ansatz soll den Verbraucher und das damit zusammenhängende Verbraucherrecht stärken. Zunächst hilft es dem Verbraucher schneller seinen entstandenen Schaden (zumindest teilweise) zurück zu erstatten. Dieser gibt seine Daten und Ansprüche bei einer Firma an, diese Zahlt ihm dann einen gewissen Prozentsatz des Streitwertes für seine Ansprüche. Der Verbraucher kommt somit schnell zu seinem Schadensersatz und muss keine weiteren Bemühungen machen. Das Risiko des Scheiterns der Geltendmachung der Ansprüche wird an die Firma abgetreten. Diese Firmen vertreten dann oft eine Vielzahl von Mandanten (beziehungsweise dann Ansprüche) in der selben Sache. Das ist auch aus Sicht der Prozessökonomie begrüßenswert. Da Gerichte mit der steigenden Automatisierung und damit erhöhten Klagewellen überlastet werden, können Legal Tech-Unternehmen die Ansprüche von vielen Verbrauchen gesammelt geltend machen und nicht für jeden einzelnen Anspruch ein eigenes Verfahren eröffnen. Die Unternehmen selbst haben ein interesse daran, Fälle die von der Norm abweichen, nicht weiter vor Gericht zu verhandeln da wie bereits erwähnt typischerweise der Streitwert gering ist \cite{Engelhardt}
\section{Prozessoptimierung}
Mit Prozessoptimierung ist eine geplante Veränderung der Prozesse, um Effektivität und Effizienz der ausgeführten Aktivitäten zu verbessern gemeint. Legal Tech zwingt Kanzleien und Unternehmen dazu, in allen internen Abläufen und Prozessen zu optimieren und standardisieren. Das ist am Markt die Voraussetzung, um weiterhin gewinnbringend Rechtsdienstleistung anbieten zu können. Aus diesem Grund fördert Legal Tech die Spezialisierung der Kanzleien auf wenige Rechtsgebiete, um möglichst hohe Übereinstimmung und damit Standardisierung zu ermöglichen. Indirekt verbessert damit Legal Tech die Rechtsdienstleistung als solche. Denn wenn Kanzleien sich nicht auf viele verschiedene Themen konzentrieren, sondern sich auf wenige Spezialisieren, werden somit Experten und damit bessere Rechtsdienstleistung geschaffen.
\section{Arbeit des Anwalts}
Durch den steigenden Legal Tech Anteil im Rechtssektor werden Juristen dazu gezwungen, sich mit dieser Technik auseinander zu setzen. Dies hat zur Folge, dass auch in der Ausbildung immer mehr darauf geachtet wird, dass Anwälte nicht nur juristisch, sondern auch technisch auf dem höchsten Niveau sind. In Deutschland wird beispielsweise an den Universitäten Hamburg, Münster als auch an der LMU München Legal Tech als Fach angeboten \cite{Engelhardt}. 
\section{In der Judikativen}
\subsection{Digitaler Schriftverkehr}
\subsection{Online-Prozesse}


\section{title}