\chapter{Einleitung}

\section{Relevanz und Ziel der Arbeit}
Seit 2017 der deutsche Anwaltstag als einer der Hauptthemen Legal Tech angeführt hat, hat das Thema für den Rechtssektor rasant an Bedeutung gewonnen. In einer Branche die laut Statista im Jahr 2018 28,82 Mrd Euro erwirtschaftet hat, spielt Technik eine immer größer werdende Rolle \footfullcite{StatistikUmsatz}. Laut des STAR Bericht der Bundesrechtsanwaltskammer 2020 sind 77\% der befragten Anwälte überzeugt, dass Legal Tech einen entscheidenden Vorteil für ihre Kanzlei bietet \footfullcite{STARStatistik}. Die Leitfrage der Arbeit lautet: Hat Legal Tech das Potential, den Markt zu verändern und falls ja, wie weit ist ein Fortbestehen der alten Methoden neben Legal Tech möglich?
\section{Methodik}

Für diese Arbeit wurde Fachliteratur über Google Scholar sowie Literatur aus der Hochschulbibliothek der Hochschule für Technik Wirtschaft und Gestaltung Konstanz verwendet. Hauptsächlich wurde mit ,, Legal Tech
und Legal Robots - Der Wandel im Rechtsmarkt
durch neue Technologien
und künstliche Intelligenz '' von Jens Wagner\footfullcite{LegalRobots} in der zweiten Auflage gearbeitet. Dieses bietet aus technischer Sicht den umfänglichsten Einblick in das Thema Legal Tech und wurde außerdem von verschiedenen Quellen zitiert und deshalb ausgewählt. Darauf aufbauend wurde die Rückwärtssuche angewandt. Hier durch, und durch Veröffentlichungen des selben Verlags wurden die meisten Quellen selektiert. Alle genutzten Quellen wurden zwischen 2018 und 2022 veröffentlicht. Dies liegt dem Fakt zu Grunde, dass Legal Tech erst seit Mitte der 2010er in Deutschland an Relevanz gewinnt. Somit wurde auch erst wenig zitierfähige Fachliteratur veröffentlicht. Die
Arbeit bietet ein Gesamtbild über das Thema Legal Tech und dient
nicht als Handlungsempfehlung. Alle getroffen Schlüsse sind lediglich persönliche Einschätzungen aufgrund des in der Literatur gewonnenen Wissens.

\section{Aufbau der Arbeit}

Zunächst wird der Begriff Legal Tech definiert und eine grobe Kategorisierung der verschiedenen Einsatzbereiche beschrieben. Das nächste Kapitel beschäftigt sich mit eben diesen Einsatzbereichen. Diese werden in wiederum verschiedene Typen eingeteilt und innerhalb dieser werden die Methoden des Typs beschrieben. Im darauf folgenden Kapitel wird auf die Auswirkung dieser Methoden eingegangen und die sich dadurch ergebende Veränderung des Marktes beschrieben. Anschließend werden die durch Legal Tech entstandenen rechtlichen Komplikationen erläutert. Abschließend wird ein Fazit und ein Ausblick in die Zukunft gegeben. 